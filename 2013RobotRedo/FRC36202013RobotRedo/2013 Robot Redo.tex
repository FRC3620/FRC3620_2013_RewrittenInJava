\documentclass[]{article}

\usepackage[left=.5in,right=.5in,bottom=.75in,top=.75in,includeheadfoot]{geometry}
%,showframe

\usepackage{layout}

\usepackage{titling}

\usepackage{color}

\definecolor{pblue}{rgb}{0.13,0.13,1}
\definecolor{pgreen}{rgb}{0,0.5,0}
\definecolor{pred}{rgb}{0.9,0,0}
\definecolor{pgrey}{rgb}{0.46,0.45,0.48}

\usepackage{listings}
\lstset{numbers=left,
	numberstyle=\tiny,
	stepnumber=5,
	numbersep=5pt,
	language=java,
	commentstyle=\color{pgreen},
	keywordstyle=\color{pblue},
	stringstyle=\color{pred},
	frame=single,
	basicstyle=\ttfamily
}

% no paragraph indent
\setlength{\parindent}{0cm}

\usepackage{enumitem}  
\SetLabelAlign{parright}{\strut\smash{\parbox[t]{\labelwidth}{\raggedleft#1}}}
\setlist[description]{style=multiline,leftmargin=3.5cm, align=parright,noitemsep}

\usepackage[yyyymmdd,hhmmss]{datetime}


%opening
\title{FIRST Team 3620\\
 2013 Robot Redo Software Specification and Design}
%\author{Your name}
\date{\today\ \currenttime}

%-----

\iftrue
\usepackage{fancyhdr}

\pagestyle{fancy}

\lhead{\thedate}
\chead{}
\rhead{\thetitle}
\lfoot{}
\cfoot{\thepage}	
\rfoot{}
% bottom line
\renewcommand{\footrulewidth}{0.4pt}

\makeatletter
\newcommand{\resetHeadWidth}{\fancy@setoffs}
\makeatother

\fi

%-----

\setcounter{tocdepth}{3}
\setcounter{secnumdepth}{4}

\raggedright

\begin{document}

\maketitle

\tableofcontents
\newpage

\section{Requirements}

\subsection{Autonomous Requirements}
Operator can pick one of these from the SmartDashboard.
\subsubsection{Box}
Robot drives in a box.

\subsubsection{Shoot3}
Robot shoots 3 preloaded discs in place.

It actuates the tilt, sets the tilt to some angle (that comes from the dashboard), then in parallel starts the shooter wheels, indexes 1.
After index is complete, it double pumps the flipper, indexes the next disk, double pumps, indexes, and double pumps.
It then turns the shooter wheels off, and sets the tilt to 0 degrees.

\subsection{Teleop Requirements}

There are 3 user interface elements on the driver's station: a driver's joystick, a shooter's joystick, and a custom control panel for manually working the disk handling elements of the robot.

\subsubsection{Driving}
The driver's joystick controls the motion of the robot on the floor.
The robot usually uses 4 of the 6 motors; if the Turbo button is depressed, all 6 motors are used.
The driver joystick also has a "reverse" button that toggles the robot between forward and reverse mode; when in reverse, the robot drives as if the rear of the robot is now the front.

\subsubsection{Lift}
There are 2 lift buttons on the driver's joystick.
While the "up" button is pressed, the lift extends, while the "down" button is pressed, the lift retracts.

\subsubsection{Shooter}
There is a button on the driver's joystick that, while held, turns the shooter on.
If the shooter is on, then the desired front speed is controlled by driver's speed up and down buttons. There are 2 speeds.
The desired and actual RPM are displayed on the dashboard.

\subsubsection{Flipper}
When the shoot button on the driver's joystick or control panel is pressed, the flipper will move out for so long, wait for so long, retract for so long, then turn off.
The control panel has a manual override toggle switch to work the flipper motor either forwards or backwards while the switch is actuated.

\subsubsection{Auger}
If the switch at the bottom of the auger is hit, then the auger will index up one disk position, unless there is a disk at the top.
There is a toggle switch on the control panel that will moves the auger either up or down while the switch is actuated.

\subsubsection{Harvester}
There is a button on the driver's joystick that toggles the harvester wheels and the belts on/off state.
There are toggle switches on the control panel that override the on/off setting for the belt and harvester wheels separately.
When the toggles are in the middle, the on/off setting prevails (when toggled on, the belt and harvester bring disks inward).

\subsubsection{ShooterTilt}
There is a button on the shooter's joystick that turns the tilt motor on and off.
If the tilt motor is on, then the desired angle is controlled by up and down buttons on the shooter's joystick.
Each depression moves the tilt 30 degrees up or down. 
The limits of travel are 0 and 60 degrees.
The actual and desired tilt are displayed on the dashboard.

\newpage

\setcounter{secnumdepth}{4}

\section {Coding Guidelines}

\subsection {Capitalization}

\begin{itemize}
\item Packages are all lower case: org.usfirst.frc3620.name (org.usfirst.frc3620.robotredo)
\item Classes are capitalized and camel-cased.
\item Methods are lower case and camel-cased.
\end{itemize}

\subsection {Naming}

\begin{itemize}
\item Projects are named FRC3620yyyyname (FRC36202013RobotRedo)
\item Subsystem classes have “Subsystem” at the end of the name: DriveSubsystem.
\item Command classes have “Command” at the end of the name, and the name of the subsystem they work with at the beginning DriveArcadeDriveCommand.
\item If a command works several subsystems, then I don't know what to put at the beginning of it's name.
\item Autonomous....
\end{itemize}

\subsection {Command-Based Programming Common Practices}
Generally, methods are put into the subsystem to write to or read from the hardware; these routines should do as much as possible to disguise the hardware (in case we need to make a hardware change). 
If a Command is written so that it using an actual hardware device (DigitalIO, Victor, whatever), then something needs to be looked at.
\begin{itemize}
\item We have a RobotMode class; it is used to indicate what mode the robot is in (disabled, autonomous, teleop, test).
This is passed into \lstinline[]|void periodic()| methods in the subsystems.
\item There is a \lstinline[]|periodic(RobotMode)| method added to the Robot.java.
The \lstinline[]|teleopPeriodic()|,  \lstinline[]|autonomousPeriodic()|,  \lstinline[]|disabledPeriodic()|, and  \lstinline[]|testPeriodic()| methods all call  \lstinline[]|periodic(RobotMode)|.
This method calls the \lstinline[]|periodic(RobotMode)| in each of the subsystems that need it, and updates the dashboard and network tables with any robot-wide information.
The subsystem's \lstinline[]|periodic(RobotMode)| methods update the dashboard and network tables with subsystem information.
\item There is a \lstinline[]|onRobotModeChange(RobotMode)| method added to the Robot.java.
The \lstinline[]|teleopInit()|,  \lstinline[]|autonomousInit()|,  \lstinline[]|disabledInit()|, and  \lstinline[]|testInit()| methods all call  \lstinline[]|onRobotModeChange(RobotMode)|.
This method calls the \lstinline[]|onRobotModeChange(RobotMode)| in each of the subsystems that need it.
Subsystems use this routine to reset any information that needs changing with the robot makes a change to a new mode (such as entering or leaving test mode).
\item Subsystems may have a constructor added to them if they need to do one-time processing, such as setting up custom Buttons that are actuated by on-robot hardware (custom Buttons that are actuated by operator interface are set up in OI.java).
\end{itemize}

\newpage

\section{Design}

\subsection{Robot}

\subsection{Autonomous Commands}

There will be two Autonomous commands: AutonomousBoxCommand and AutonomousShoot3Command.

\subsection{DriveSubsystem}

The default command will be the DriveArcadeCommand command.

\subsubsection{Actuators, Controllers, and Sensors}

\textbf{PrimaryDrive} is the drive for 4 of the 6 drive motors.

\textbf{SecondaryDrive} is the drive for the remaining 2 drive motors.

\textbf{Gyro} is the gyro used to measure turn.

\textbf{DriveEncoder} is the encoder used to measure distance traveled.

\subsubsection{Commands}

\paragraph{DriveArcadeCommand} should be the default command for the subsystem.
\begin{description}[topsep=0ex]
\item[requires] DriveSubsystem
\item[initialization]  None.
\item[execute] Uses the arcadeDrive method to drive the robot.
\item[isDone] Runs until interrupted.
\item[end] Turns off drive.
\item[interrupted] Same as end();
\end{description}

\paragraph{DriveTurnInPlaceCommand(double \textit{degrees})} is used to turn \textit{degrees} degrees. Negative values are to the left, positive values are to the right.
\begin{description}[topsep=0ex]
\item[requires] DriveSubsystem
\item[initialization]  use resetGyro()
\item[execute] Uses the arcadeDrive method to drive the robot.
\item[isDone] finishes when the robot has turned far enough.
\item[end] Turns off drive.
\item[interrupted] Same as end();
\end{description}

\paragraph{DriveMoveInLineCommand(double \textit{meters})} is used to move \textit{meters} meters. Positive values are forward, negative values are backward.
\begin{description}[topsep=0ex]
\item[requires] DriveSubsystem
\item[initialization]  use resetEncoder()
\item[execute] Uses the arcadeDrive method to drive the robot.
\item[isDone] finishes when the robot has moved far enough.
\item[end] Turns off drive.
\item[interrupted] Same as end();
\end{description}

\paragraph{DriveToggleReverseCommand()} toggles "reverse mode".
\begin{description}[topsep=0ex]
\item[requires] Nothing
\item[initialization] None
\item[execute] use the subsystem's "toggle reverse" method.
\item[isDone] Runs once.
\item[end] Turns off drive.
\item[interrupted] Same as end();
\end{description}

\subsubsection{Public Methods}

\noindent
\lstinline[]|public void arcadeDrive (GenericHID hid, boolean turbo)| \\
drives the robot using just the PrimaryDrive if turbo is false, or both PrimaryDrive and SecondaryDrive if turbo is true.
The robot drives in reverse if reverse mode is toggled on.

\noindent
\lstinline[]|public void drive (double outputMagnitude, double curve)| \\
uses the PrimaryDrive to drive the robot. 
This is for use by autonomous modes.

\noindent
\lstinline[]|public void halt ()| \\
stops all drive motors.

\noindent
\lstinline[]|public void resetGyro ()| \\
resets the gyro.

\noindent
\lstinline[]|public double readGyro ()| \\
reads the gyro heading in degrees.

\noindent
\lstinline[]|public void resetEncoder ()| \\
resets the encoder.

\noindent
\lstinline[]|public double readEncoder ()| \\
reads the distance traveled in meters.

\noindent
\lstinline[]|public void toggleReverseMode ()| \\
toggles the reverse mode flag.

\noindent
\lstinline[]|public void setReverseMode (boolean reverse)| \\
sets the reverse mode flag to \textit{reverse}.

\noindent
\lstinline[]|public boolean getReverseMode()| \\
returns the current value of the reverse modem flag.

%------------------------------------------------------------

\subsection{LiftSubsystem}

\subsubsection{Actuators, Controllers, and Sensors}

\textbf{ChinupController} is the Victor controller for the chinup motors.

\subsubsection{Commands}

\paragraph{LiftExtendCommand} extends the lift.
\begin{description}[topsep=0ex]
\item[requires] LiftSubsystem
\item[initialization]  None.
\item[execute] Uses the liftExtend method to extend the lift.
\item[isDone] Runs until interrupted.
\item[end] liftHalt()
\item[interrupted] Same as end();
\end{description}

\paragraph{LiftRetractCommand} retracts the lift.
\begin{description}[topsep=0ex]
\item[requires] LiftSubsystem
\item[initialization]  None.
\item[execute] Uses the liftRetract method to retract the lift.
\item[isDone] Runs until interrupted.
\item[end] liftHalt()
\item[interrupted] Same as end();
\end{description}

\subsubsection{Public Methods}

\noindent
\lstinline[]|public void liftRetract()| \\
retracts the lift (lifts the robot).

\noindent
\lstinline[]|public void liftExtend()| \\
extends the lift (lowers the robot).

\noindent
\lstinline[]|public void liftHalt()| \\
freezes the lift.

%------------------------------------------------------------

\subsection{ShooterSubsystem}

The front shooter runs at either 800 or 1600 RPM.

\subsubsection{Actuators, Controllers, and Sensors}

\textbf{FrontShooterController} is the Victor controller for the front shooter motor.

\textbf{ReadShooterController} is the Victor controller for the rear shooter motor.

\textbf{FrontShooterCounter} is the Counter for the front shooter motor.

\textbf{FrontPID} is the PID controller for the front shooter motor.

\subsubsection{Commands}

\paragraph{ShooterRunRearShooterCommand} is the default command for the subsystem, it keeps the rear motor power set. The front motor power does not need to be set here because the FrontPid PIDController keeps the front motor power set.
\begin{description}[topsep=0ex]
\item[requires] ShooterSubSystem
\item[initialization]  None.
\item[execute] Send the desired motor power (getDesiredRearMotorPower()) to the rear shooter motor (setRearMotorPower())
\item[isDone] Runs until interrupted.
\item[end] shuts down the rear shooter motor (setRearMotorPower()).
\item[interrupted] Same as end();
\end{description}

\paragraph{ShooterFasterCommand} bumps the desired RPM for the front shooter.
\begin{description}[topsep=0ex]
\item[requires] nothing (we don't want to interrupt any commands)
\item[initialization]  None.
\item[execute] Bumps the desired RPM up.
\item[isDone] Runs once.
\item[end] nothing
\item[interrupted] Same as end();
\end{description}

\paragraph{ShooterSlowerCommand} bumps the desired RPM for the front shooter.
\begin{description}[topsep=0ex]
\item[requires] nothing (we don't want to interrupt any commands)
\item[initialization]  None.
\item[execute] Bumps the desired RPM down.
\item[isDone] Runs once.
\item[end] nothing
\item[interrupted] Same as end();
\end{description}

\paragraph{ShooterButtonCommand} turns the shooters on while running.
\begin{description}[topsep=0ex]
\item[requires] nothing (we don't want to interrupt any commands)
\item[initialization]  None.
\item[execute] \emph{needs work}
\item[isDone] Runs until interrupted.
\item[end] \emph{needs work}
\item[interrupted] Same as end();
\end{description}

\paragraph{ShooterOnCommand} turns the shooters on.
\begin{description}[topsep=0ex]
\item[requires] nothing (we don't want to interrupt any commands)
\item[initialization]  None.
\item[execute] setRearShooterPower(), enableFrontPID()
\item[isDone] Runs once.
\item[end] nothing
\item[interrupted] Same as end();
\end{description}

\paragraph{ShooterOffCommand} turns the shooters on.
\begin{description}[topsep=0ex]
\item[requires] nothing (we don't want to interrupt any commands)
\item[initialization]  None.
\item[execute] setRearShooterPower(), disableFrontPID()
\item[isDone] Runs once.
\item[end] nothing
\item[interrupted] Same as end();
\end{description}

\subsubsection{Public Methods}

\noindent
\lstinline[]|public void setDesiredFrontRPM (double rpm)| \\
sets the desired speed for the front shooter motor.

\noindent
\lstinline[]|public double getDesiredFrontRPM ()| \\
gets the desired speed for the front shooter motor.

\noindent
\lstinline[]|public void setDesiredRearMotorPower (double power)| \\
sets the desired power for the rear shooter motor (saves the desired power).

\noindent
\lstinline[]|public double getDesiredRearMotorPower ()| \\
gets the desired power for the rear shooter motor.

\noindent
\lstinline[]|public double getCurrentFrontRPM ()| \\
gets the current (actual) speed for the front shooter motor.

\noindent
\lstinline[]|public void enableFrontPID ()| \\
starts the PID controller for the front motor.

\noindent
\lstinline[]|public void disableFrontPID ()| \\
stops the PID controller for the front motor.

\noindent
\lstinline[]|public void setRearShooterPower(double power)| \\
set the power for rear shooter motor. Actually pokes the motor controller.

\noindent
\lstinline[]|public setFrontShooterPower(double power)| \\
sets the power for front shooter motor. Actually pokes the motor controller. If the specified power is greater than 0.5, cap it at 0.5.

%------------------------------------------------------------

\subsection{FlipperSubsystem}

\subsubsection{Actuators, Controllers, and Sensors}

\textbf{FlipperSpike} is the Spike controller for the flipper motor.

\subsubsection{Commands}

\paragraph{FlipperFlipCommand} extends the lift.
\begin{description}[topsep=0ex]
\item[requires] FlipperSubsystem
\item[initialization]  save the current time.
\item[execute] Uses flipperForward() to move the flipper forward for so long, flipperHalt() for so long, flipperBackward() for so long.
\item[isDone] Runs until the flipperBackward at the end is done.
\item[end] flipperHalt()
\item[interrupted] Same as end();
\end{description}

\paragraph{FlipperAutonomousFlipCommand} pumps the flipper multiple times.
\begin{description}[topsep=0ex]
\item[requires] FlipperSubsystem
\item[initialization]  save the current time.
\item[execute] \emph{needs work}
\item[isDone] Runs until the flipperBackward at the end is done.
\item[end] flipperHalt()
\item[interrupted] Same as end();
\end{description}

\paragraph{FlipperForwardCommand} moves the flipper forward while running.
\begin{description}[topsep=0ex]
\item[requires] FlipperSubsystem
\item[initialization]  None.
\item[execute] Uses flipperForward() to move the flipper forward.
\item[isDone] Runs until interrupted.
\item[end] flipperHalt()
\item[interrupted] Same as end();
\end{description}

\paragraph{FlipperBackwardCommand} moves the flipper backwards while running.
\begin{description}[topsep=0ex]
\item[requires] FlipperSubsystem
\item[initialization]  None.
\item[execute] Uses flipperBackward() to move the flipper backward.
\item[isDone] Runs until interrupted.
\item[end] flipperHalt()
\item[interrupted] Same as end();
\end{description}

\subsubsection{Public Methods}

\noindent \texttt{\textbf{public void flipperForward()}} \\
moves the flipper forward.

\noindent \texttt{\textbf{public void flipperBackward()}} \\
moves the flipper backward.

\noindent \texttt{\textbf{public void flipperHalt()}} \\
turns the flipper motor off.

%------------------------------------------------------------

\subsection{AugerSubsystem}

\subsubsection{Actuators, Controllers, and Sensors}

\textbf{AugerController} is the Victor controller for the auger motor. \textit{Chris: check the old robot; this could be a Spike. If so, make the change here.}

\textbf{AugerLimitSwitch} is the digital input for the limit switch.

\textbf{AugerEncoder} is the position encoder for the auger.

\subsubsection{Commands}

\paragraph{AugerIndexCommand} indexes the auger one frisbee.
\begin{description}[topsep=0ex]
\item[requires] AugerSubsystem
\item[initialization] Record the current time.
\item[execute] Uses the augerUp method to move the frisbees.
\item[isDone] Runs when then auger is indexed and the elapsed time is more than 0.5 seconds.
\item[end] Turns off drive (augerHalt())
\item[interrupted] Same as end();
\end{description}

\paragraph{AugerUpCommand} moves the auger in the up direction as long as it runs.
\begin{description}[topsep=0ex]
\item[requires] AugerSubsystem
\item[initialization]  None.
\item[execute] Uses the augerUp method to move the frisbees.
\item[isDone] Runs until interrupted.
\item[end] Turns off drive augerHalt())
\item[interrupted] Same as end();
\end{description}

\paragraph{AugerDownCommand} moves the auger in the down direction as long as it runs.
\begin{description}[topsep=0ex]
\item[requires] AugerSubsystem
\item[initialization]  None.
\item[execute] Uses the augerDown method to move the frisbees.
\item[isDone] Runs until interrupted.
\item[end] Turns off drive augerHalt())
\item[interrupted] Same as end();
\end{description}

\subsubsection{Public Methods}

\noindent
\lstinline[]|public boolean readAugerLimitSwitch()| \\
reads the limit switch at the bottom of the auger.

\noindent
\lstinline[]|public boolean isAugerNeutral()| \\
tells if the auger is in the neutral position.

\noindent
\lstinline[]|public void augerUp()| \\
turn the auger motor so that frisbees move to the top.

\noindent
\lstinline[]|public void augerDown()| \\
turn the auger motor so that frisbees move to the bottom.

\noindent
\lstinline[]|public void augerHalt()| \\
turn the auger motor off.

\subsubsection{Internal Methods}

\noindent
\lstinline[]|double readAugerEncoder()| \\
reads the auger position in degrees.
I'll bet that the code to update the dashboard and isAugerNeutral() use this.

%------------------------------------------------------------

\subsection{HarvesterSubsystem}

\subsubsection{Actuators, Controllers, and Sensors}

\textbf{WheelsController} is the Victor controller for the harvester wheel motors.

\textbf{BeltController} is the Spike for the belt motor.

\subsubsection{Commands}

\paragraph{HarvesterRunCommand} keeps the harvester and belt motors at the proper speed.
\begin{description}[topsep=0ex]
\item[requires] HarvesterSubsystem
\item[initialization] None.
\item[execute] \emph{needs work}
\item[isDone] Runs until interrupted.
\item[end] \emph{needs work}
\item[interrupted] Same as end();
\end{description}

\paragraph{HarvesterWheelsInManualCommand} runs while the control panel HarvesterWheel override is set to move the wheels inward.
\begin{description}[topsep=0ex]
\item[requires] None.
\item[initialization]  None.
\item[execute] \emph{needs work}
\item[isDone] Runs until interrupted.
\item[end] \emph{needs work}
\item[interrupted] Same as end();
\end{description}

\paragraph{HarvesterWheelsOutManualCommand} runs while the control panel HarvesterWheel override is set to move the wheels outward.
\begin{description}[topsep=0ex]
\item[requires] None.
\item[initialization]  None.
\item[execute] \emph{needs work}
\item[isDone] Runs until interrupted.
\item[end] \emph{needs work}
\item[interrupted] Same as end();
\end{description}

\paragraph{HarvesterBeltInManualCommand} runs while the control panel Belt override is set to move the belt inward.
\begin{description}[topsep=0ex]
\item[requires] None.
\item[initialization]  None.
\item[execute] \emph{needs work}
\item[isDone] Runs until interrupted.
\item[end] \emph{needs work}
\item[interrupted] Same as end();
\end{description}

\paragraph{HarvesterBeltOutManualCommand} runs while the control panel Belt override is set to move the belt outward.
\begin{description}[topsep=0ex]
\item[requires] None.
\item[initialization]  None.
\item[execute] \emph{needs work}
\item[isDone] Runs until interrupted.
\item[end] \emph{needs work}
\item[interrupted] Same as end();
\end{description}

\paragraph{HarvesterToggleCommand} toggles whether the harvester is on or off.
\begin{description}[topsep=0ex]
\item[requires] None.
\item[initialization]  None.
\item[execute] \emph{needs work}
\item[isDone] Runs once.
\item[end] \emph{needs work}
\item[interrupted] Same as end();
\end{description}

\subsubsection{Public Methods}

There is a HarvesterDirection Java class that represents the desired direction for the Harvester components.
A HarvesterDirection can have three values HarvesterDirection.IN, HarvesterDirection.OUT, HarvesterDirection.OFF.

\noindent
\lstinline[]|public void setOnOffState(boolean b)| \\
saves the current on/off state of the joystick button.

\noindent
\lstinline[]|public boolean getOnOffState()| \\
gets the current on/off state of the joystick button.

\noindent
\lstinline[]|public void toggleOnOffState()| \\
toggles the current on/off state.

\noindent
\lstinline[]|public void setBeltDirection (HarvesterDirection direction)| \\
saves the desired manual belt direction.

\noindent
\lstinline[]|public HarvesterDirection getBeltDirection ()| \\
gets the desired manual belt direction.

\noindent
\lstinline[]|public void setWheelDirection (HarvesterDirection direction)| \\
saves the desired harvester wheel direction.

\noindent
\lstinline[]|public HarvesterDirection getWheelDirection ()| \\
gets the desired manual wheel direction.

\noindent
\lstinline[]|public void wheelsOff()| \\
turns the wheel motors off.

\noindent
\lstinline[]|public void wheelsIn()| \\
runs the wheel motors so that they bring disks in.

\noindent
\lstinline[]|public void wheelsOut()| \\
runs the wheel motors so that they spew disks out.

\noindent
\lstinline[]|public void beltOff()| \\
turns the belt motor off.

\noindent
\lstinline[]|public void beltIn()| \\
runs the belt motor so that it brings disks in.

\noindent
\lstinline[]|public void beltOut()| \\
runs the belt motor so that it spews disks out.
%------------------------------------------------------------

\subsection{ShooterTiltSubsystem}

\subsubsection{Actuators, Controllers, and Sensors}

\textbf{ShooterTiltController} is the Victor controller for the shooter tilt motor.

\textbf{ShooterTiltSensor} is the analog input for the shooter tilt potentiometer.

\textbf{ShooterTiltPID} is the PID controller for the shooter tilt.

\subsubsection{Commands}

\paragraph{ShooterTiltBumpAngleUpCommand} bumps the desired shooter angle up.
\begin{description}[topsep=0ex]
\item[requires] None.
\item[initialization] None.
\item[execute] \emph{needs work}
\item[isDone] Runs once.
\item[end] \emph{needs work}
\item[interrupted] Same as end();
\end{description}

\paragraph{ShooterTiltBumpAngleDownCommand} bumps the desired shooter angle down.
\begin{description}[topsep=0ex]
\item[requires] None.
\item[initialization] None.
\item[execute] \emph{needs work}
\item[isDone] Runs once.
\item[end] \emph{needs work}
\item[interrupted] Same as end();
\end{description}

\paragraph{ShooterTiltEnableCommand} enables the shooter tilt.
\begin{description}[topsep=0ex]
\item[requires] None.
\item[initialization] None.
\item[execute] \emph{needs work}
\item[isDone] Runs once.
\item[end] \emph{needs work}
\item[interrupted] Same as end();
\end{description}

\paragraph{ShooterTiltToggleCommand} toggles the enabled/disabled state of the shooter tilt.
\begin{description}[topsep=0ex]
\item[requires] None.
\item[initialization] None.
\item[execute] \emph{needs work}
\item[isDone] Runs once.
\item[end] \emph{needs work}
\item[interrupted] Same as end();
\end{description}

\paragraph{ShooterTiltDisableCommand} disables the shooter tilt.
\begin{description}[topsep=0ex]
\item[requires] None.
\item[initialization] None.
\item[execute] \emph{needs work}
\item[isDone] Runs once.
\item[end] \emph{needs work}
\item[interrupted] Same as end();
\end{description}

\subsubsection{Public Methods}

\emph{This needs work.}

\noindent
\lstinline[]|public void enableShooterTilt()| \\
enables the shooter tilt.

\noindent
\lstinline[]|public void disableShooterTilt()| \\
disables the shooter tilt.

\noindent
\lstinline[]|public void toggleShooterTilt()| \\
toggles the shooter tilt.

\noindent
\lstinline[]|public void setDesiredShooterTiltAngle(double degrees)| \\
set (saves) the desired shooter tilt angle.

\noindent
\lstinline[]|public double getDesiredShooterTiltAngle()| \\
gets the saved shooter tilt angle.

\noindent
\lstinline[]|public double getCurrentShooterTiltAngle()| \\
reads the current shooter tilt angle.

%------------------------------------------------------------

\subsection{Operator Interface}

\begin{tabular}{|l|l|l|l|l|}
\hline \emph{Name} & \textbf{Joystick} & Button Number & Command & When to Run \\ 
\hline LiftExtend & Driver Joystick & 6 & LiftExtendCommand & whileHeld  \\ 
\hline LiftRetract & Driver Joystick & 7 & LiftRetractCommand & whileHeld  \\ 
\hline DriveTurboEnable & Driver Joystick & 3 & (none) & (none)  \\ 
\hline DriveDirectionToggle & Driver Joystick & 8 & (none) & (none)  \\ 
\hline HarvesterOnOff & Driver Joystick & 1 & HarvesterToggleCommand & whenPressed  \\ 
\hline ShooterOn & Driver Joystick & ? & ShooterOnCommand & whileHeld  \\ 
\hline FlipperButton2 & Control Panel & 1 & FlipperFlipCommand & whenPressed  \\ 
\hline FlipperForward & Control Panel & 8 & FlipperForwardCommand & whileHeld  \\ 
\hline FlipperBackward & Control Panel &7? & FlipperBackwardCommand & whileHeld  \\ 
\hline AugerUp & Control Panel & 10 & AugerUpCommand & whileHeld  \\ 
\hline AugerDown & Control Panel & 9 & AugerDownCommand & whileHeld  \\ 
\hline HarvesterIn & Control Panel & 6 & HarvesterWheelsInManualCommand & whileHeld  \\ 
\hline HarvesterOutt & Control Panel & 5 & HarvesterWheelsOutManualCommand & whileHeld \\ 
\hline BeltIn & Control Panel & 4 & HarvesterBeltInManualCommand & whileHeld  \\ 
\hline BeltOut & Control Panel & 3 & HarvesterBeltOutManualCommand & whileHeld \\ 
\hline FlipperButton & Shooter Joystick & 1 & FlipperFlipCommand & whenPressed  \\ 
\hline ShooterOn & Shooter Joystick & 4 & ShooterOnCommand & whileHeld  \\ 
\hline ShooterFaster & Driver Joystick & 3 & ShooterFasterCommand & whenPressed  \\ 
\hline ShooterSlower & Driver Joystick & 2 & ShooterSlowerCommand & whenPressed  \\ 
\hline AugerIndex & Shooter Joystick & 5 & AugerIndexCommand & whenPressed  \\ 
\hline ShooterTiltOff & Shooter Joystick & 8 & ShooterTiltDisableCommand & whenPressed  \\ 
\hline ShooterTiltUp & Shooter Joystick & 7 & ShooterTiltBumpAngleUpCommand & whenPressed  \\ 
\hline ShooterTiltDown & Shooter Joystick & 1\emph{(!)} & ShooterTiltBumpAngleDownCommand & whenPressed  \\ 
\hline 
\end{tabular} 

\newpage

\appendix

\section{RobotMode.java}

%\lstset{basicstyle=\listingsfontinline}

\begin{lstlisting}
public class RobotMode {
   private String name;
   public RobotMode (String n) {
       name = n;
   } 
   public String toString() {
       return name;
   }
   public static final RobotMode DISABLED = new RobotMode("DISABLED");
   public static final RobotMode AUTONOMOUS = new RobotMode("AUTONOMOUS");
   public static final RobotMode TELEOP = new RobotMode("TELEOP");
   public static final RobotMode TEST = new RobotMode("TEST");
}
\end{lstlisting}

\newpage

\layout

\end{document}